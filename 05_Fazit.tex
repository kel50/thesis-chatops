\chapter{Fazit} \label{Fazit}

\section{Zusammenfassung und Schlussfolgerung}
ChatOps zur Bedienung der \acs{CMDB} hat sich als neuartiges Thema herausgestellt, das noch schwer greifbar ist, weswegen auch keine reine Literaturarbeit möglich war, sondern Experteninterviews durchgeführt wurden.

Die Experten haben allesamt großes Potential in ChatOps im Allgemeinen gesehen und konnten sich für den konkreten Anwendungsfall \acs{CMDB} einige Funktionen bzw. Tätigkeiten  vorstellen, die von einem Chatbot übernommen oder zumindest unterstützt werden können.

Auch die Rahmenbedingungen, die gesetzt sein müssen, damit ein \acs{CMDB} Chatbot betrieben werden kann, konnten anhand der Experteninterviews in Form von Qualitätsanforderungen erarbeitet werden. Wichtig dabei war es vor Allem die spezielle Perspektive \textit{Rechenzentrum} zu betrachten, da für Mitarbeiter, die in dieser Branche arbeiten Aspekte wie beispielsweise die Betriebssicherheit sehr wichtig sind. %Besonders in der Wichtigkeit der Betriebssicherheit hat sich diese Ausprägung gezeigt. 

Ob die Einführung eines \acs{CMDB} Chatbots wirklich zu einer Arbeitserleichterung führt, kann sich nur in der Praxis zeigen, da es bisher noch keine dokumentierten Referenzvorhaben gibt, ChatOps im Rechenzentrumsumfeld und speziell zur Bedienung der \acs{CMDB} einzusetzen. Mit der Befragung der Experten wurde jedoch schon ein wichtiger Schritt getan, um die Voraussetzungen dafür zu schaffen, dass das im Rahmen dieser Arbeit erstellte Chatbot Plugin für die \acs{CMDB} bei der täglichen Arbeit im Rechenzentrum wirklich eine Arbeitserleichterung bringt.

Auch die \acl{CAMS}, die ChatOps mit sich bringt und welche eine hohe Transparenz bedeutet, wird sich in der Praxis behaupten müssen. Die befragten Experten sahen zwar kein Problem in der Einführung dieser Arbeitsweise, allerdings ist das noch kein Garant für das Gelingen der Umsetzung. Es wird sich zeigen müssen, wie gut diese von den Mitarbeitern angenommen wird. Für einige bedeutet ChatOps wahrscheinlich einen Bruch mit alten Gewohnheiten und eine Umstellung der eigenen Arbeitsweise.\\
Bei mangelnder Akzeptanz ließe sich auch die Transparenz vermeiden und Befehle nur per direktem Chat mit dem Bot absetzen. Technisch ist das durchaus möglich. Wie im Praxiskapitel beschrieben wurde, lässt sich die Konfiguration dahingehend anpassen. Allerdings bedeutet ein Verzicht auf Transparenz auch einen Verzicht auf die Mehrwerte von ChatOps.

\section{Ausblick}
Bereits in den Interviews hat sich gezeigt, dass sich die Experten ChatOps durchaus auch für andere Bereiche des Rechenzentrumsbetriebs vorstellen können, bzw. gar die Vision einer Verkettung verschiedener Chatbot Plugins haben, in der ein gesamter Prozess abgebildet wird. Konkret ist das z. B.~das Anlegen eines neuen Servers. Dieser kann bereits jetzt per Chatbot in der \acs{CMDB} geplant werden und es wäre denkbar auch die DNS Einträge so vorzunehmen und eine virtuelle Maschine aus einer Vorlage zu klonen.\\
All die beteiligten Tools dieses Prozesses verfügen über eine \acs{API}. Wenn also das \acs{CMDB} Chatbot Plugin die erwartete Arbeitserleichterung bringt und auf die nötige Akzeptanz stößt, wäre es durchaus denkbar in Zukunft auch Chatbot Plugins für diese Tools oder auch weitere, die im Rahmen dieser Arbeit noch nicht genannt wurden, zu implementieren. 



\todo{Statistik}
