\chapter{Experteninterview Hemmerich}\label{Interview_Hemmerich}

Kevin Weiss (\textbf{KW}) führte mit Alexander Hemmerich (\textbf{AH}) ein Experteninterview  durch, um Anforderungen an einen CMDB Chatbot zu ermitteln. Herr Hemmerich ist Leiter des IVZ Rechenzentrums am Standort Köln.

\section*{CMDB im Rechenzentrum}

\begin{list}{X:}{\setlength{\labelsep}{5mm}}
\item[KW:] Was verstehst du unter einer CMDB im Allgemeinen?
\item[AH:] Die CMDB, die wir im Einsatz haben nutzen wir für unsere Configuration Items im Rechenzentrum, vom Kabel über Server über Ausstattung innerhalb eines Rechnerraums. Das ist so eine große Anzahl von Komponenten: Lizenzen, Supportverträge. Die lässt sich gar nicht mehr anders handhaben und verwalten. Hinzu kommt, dass sie durch unsere ISO 27001 Zertifizierung sogar gefordert ist.
\item[KW:] Wie benutzt du denn die CMDB bei der täglichen Arbeit?
\item [AH:] Um Gegenstände, und Bestände abzufragen. Also: wie viele Kabel haben wir noch im Lager, wo ist der Server XY, ist der schon verschrottet, ist der eingelagert? Und auch für die Abrechnung mit unseren Kunden, die z. B. eine Anwendung, ein Service belegt, sei es Speicher, sei es CPU, um das zusammenzutragen und für die Abrechnung an unsere kaufmännische Leitung weiterzugeben. 
\item [KW:] Beschreibe doch mal bitte einen konkreten Anwendungsfall, wie du eine CMDB nutzt, wie du einen Workflow durchläufst.
\item [AH:] Es kann vorkommen, dass ein Kunde sich zusätzliche virtuelle Maschinen wünscht. Wir prüfen dann über die CMDB, erstellen einen Report, schauen dann, welche Ressourcen sein bei uns bestellter Service zurzeit belegt, vergleichen das dann mit der Menge, die er bestellt hat und wenn da noch Luft ist, z. B. für CPU oder RAM Ressourcen kann man entsprechend der Anforderung eine Freigabe geben und die kann dann in die Umsetzung gebracht werden. Wenn die Ressourcen erschöpft sind, geht das ganze natürlich dann zu unserem Vertrieb und daraus entsteht in der Regel eine Preisinformation für den Kunden und er kann sich dann überlegen, ob er diese Ressourcen zusätzlich bestellt oder nicht.
\item [KW:] Ok und welche CMDB Funktionen findest du besonders wichtig? Also auch im Zusammenhang mit anderen Tools, die wir einsetzen? 
\item [AH:] Wir haben uns dazu committet, dass die CMDB bei uns im Haus das führende System ist. Das heißt, dass alle anderen Systeme, sei es das Monitoring oder seien es irgendwelche Scanner, die automatisch nach Systemen scannen, dass dies zuliefernde Systeme sind und die CMDB eben als führendes System allein diese Daten aggregriert, um eine Gesamtsicht auf das Rechenzentrum, auf die Ausstattung und sogar auf die Standorte des Unternehmens zu bieten. Also da sind nicht nur Rechenzentrums Configuration Items drin, sondern auch die Gebäude, die Büros, die Mitarbeiter, die Telefone usw..
\end{list}

\section*{Chatbots allgemein}

\begin{list}{X:}{\setlength{\labelsep}{5mm}}
\item [KW:] Was verstehst du unter den Begriffen Chatbots und ChatOps?
\item [AH:] Das sind zusammengesetzte Worstücke, also Bot und Chat. Ich nehme mal an, dass das Bots sind, mit denen man kommunizieren kann, die dann entweder Informationen liefern, die man angefragt hat oder vielleicht sogar eine Aktion ausführen, im definierten Umfang. 
\item [KW:] Hattest du schon mit Chatbots zu tun und welche Erfahrungen hast du dabei gemacht?
\item [AH:] Im professionellen Umfeld hatte ich da noch keine Erfahrungen mit. Ich habe allerdings im Rahmen deiner Entwicklungen im Slack die Bots gesehen, die du getestet hast. Also Erfahrung wäre übertrieben. Ich finde die Idee gut und ich bin gespannt, was man damit noch alles machen kann.
\item [KW:] Welche Vorteile könnten wir aus ChatOps ziehen?
\item [AH:] Also ich denke, dass Chatbots nicht den Menschen ersetzen werden, sondern ihn unterstützen in der täglichen Arbeit. Ein Chatbot könnte einfache Tätigkeiten übernehmen und zur Automatisierung des täglichen Rechenzentrumsbetriebs beitragen. Also man könnte z. B. ,wenn man eine Information aus der CMDB braucht den Bot fragen, welche RAM-Ressourcen in unserem Rechenzentrum die deutsche Welle belegt. Er liefert als Suchmaschine halt eine Information innerhalb des Rechenzentrums zurück, er kann aus meiner Sicht aber auch als Operator einfache Tätigkeiten wie die Erweiterung eines Filesystems oder eine DNS-Eintragung anstelle eines Admins erledigen, um den Alltag und die Automatisierung im Rechenzentrum voranzutreiben.
\end{list}

\section*{Chatbots zur Bedienung der CMDB}

\begin{list}{X:}{\setlength{\labelsep}{5mm}}
\item [KW:] Was denkst du, welche Aufgaben ein CMDB Chatbot gut übernehmen kann und welche weniger gut?
\item [AH:] Ich denke das hängt stark davon ab, welche Schnittstellen die eingesetzte CMDB bietet und welche Kommandos sie extern über diese Schnittstellen akzeptiert. Ich könnte mir gut vorstellen, dass mit der i-doit CMDB z. B. über einen Chatbot die nächste freie IP in einem Netz anzeigen lässt, dass man Objekte sucht, die es in der CMDB gibt,  um eine Abfrage zu machen: \textit{Ist dieses Gerät in der CMDB eingetragen?}. Das wären die ersten Schritte, die ich mir vortellen könnte.
\item [KW:] Also welche Funktionen würde ein Chatbot vereinfachen? Ich denke an Aktionen wie das Anlegen einer VM und das Heraussuchen einer freien IP. Denkst du, da kann man viel automatisieren?
\item [AH:] Es wäre denkbar im Vorhinein einen Server in der CMDB über den Bot zu planen, also den Servernamen einzutragen, eine freie IP in einem entsprechenden Netz zu reservieren und dann als Prozess direkt in Puppet die Konfiguration vorzunehmen, die VM auszurollen und die Netzkonfiguration herzustellen. Ich denke nur so ist zu verhindern, dass es entsprechende Inkonsistenzen gibt zwischen der produktiven Landschaft und den Einträgen in der CMDB.
\item [KW:] Welche Anforderungen hast du an einen CMDB Chatbot, wenn so etwas eingeführt wird?
\item [AH:] Es ist natürlich wichtig, dass zu jeder Zeit die Integrität der CMDB besteht und geschützt wird. Es darf nicht passieren, dass durch die Fehlfunktion oder Fehlbedienung eines Bots Daten in der CMDB entfernt oder verfälscht werden. Ich denke da muss man um die Betriebssicherheit im Rechenzentrum zu gewährleisten, bei der Entwicklung einiges bedacht werden, dass nicht jemand Fremdes unsere CMDB Integrität beeinträchtigen kann.
\item [KW:] Also sollte es nachvollziehbar sein, wer Änderungen gemacht hat und die Berechtigungen sollten nur für einen gewissen Personenkreis gelten?
\item [AH:] Das auf jeden Fall.
\item [KW:] Denkst du es könnte beim Einsatz von Chatbots Probleme geben?
\item [AH:] Ja, kann ich mir vorstellen. Allerdings besteht der größte Teil unserer Mannschaft aus jungen, motivierten Kollegen, die auch in der Freizeit viel mit neuen Technologien arbeiten. Ich könnte mir vorstellen, dass das in dem Bereich sehr großen Anklang finden wird. Andere, die schon ein paar Jahrzente in der IT sind müssen sich da vielleicht noch dran gewöhnen und nutzen dann noch die manuellen Eingaben in die CMDB. Aber ich glaube schon, dass die Bots, je nach dem wie die sie gestaltet sind, angenommen werden.
\item [KW:] Ein Grundsatz von ChatOps ist die Transparenz, also dass jeder sieht, was die anderen im Chatraum machen. Denkst du das wird im Team gut angenommen?
\item [AH:] Ja, ich glaube nicht, dass es unter den Kollegen zu Problemen kommen wird, ich glaube eher, dass man das über Betriebsräte usw. vorher abklären muss. Da gibt es eventuell eine Mitbestimmungspflicht, weil es ist ja eine Art Arbeitsüberwachung. Also der eine Kollege könnte sehen, was der andere macht, bzw. wie intensiv er an einem Tag gearbeitet hat, aber ich glaube nicht, dass es aus dem Team her zu Problemen führt.
\item [KW:] Ein Bot kann je nach Bedarf weiterentwickelt werden. Es gibt verschiedene Frameworks mit den Sprachen Ruby, Python oder Coffeescript, was denkst du wäre die sinnvollste Variante zur internen Weiterentwicklung?
\item [AH:] Wir haben aktuell einige Kolegen, die im Python Umfeld großes Know-How haben, ich empfehle immer Entwicklungen in den Sprachen zu betreiben, wo das Know-How da ist und auch in Zukunft da sein wird. Denn es gibt nichts Schlimmeres als eine Anwendung, die ein Praktikant oder Azubi entwickelt hat in einer Sprache, mit der sich hier keiner auskennt und die nach seinem Weggang eben nicht mehr wartbar ist. In dem Fall würde sich jetzt Python prima anbieten.
\item [KW:] Ein Chatbot soll auch immer eine gewisse Persönlichkeit haben. Was kannst du dir da vorstellen, also z. B. eine Figur aus Film und Fernsehen?
\item [AH:] Es gibt viele Möglichkeiten, aber ich denke da wir in einem professionellen Rechenzentrumsbetrieb sind, muss es schon etwas Ernsthaftes sein. Da wir für die Medien arbeiten, könnte es vielleicht sogar die Maus oder der Elefant sein von der Sendung mit der Maus. Natürlich sorgen wir immer dafür, dass eine gewisse Humor-Stimmung hier im Alltag herrscht, ich könnte mir auch vorstellen, dass man da ein Bisschen was Humorvolles einbaut, wenn die Anfrage etwas Humor beinhaltet, dass der Bot dann auch humorvoll antwortet. Aber natürlich dürfen wir nicht vergessen, dass es ein Werkzeug für den professionellen Einsatz sein soll und es auch äußerlich so aussehen sollte.
\item [KW:] Es wäre also schon denkbar, dass die Maus oder ein entsprechender Avatar freche Antworten gibt, wenn man die Antworten falsch formuliert oder Ähnliches? 
\item [AH:] Ja, natürlich. So ein Charakterzug würde das Ganze ein Wenig aufpeppen.
\item [KW:] Es gibt verschiedene Ansätze für Chatbots, der eine ist regelbasiert, der andere selbstlernend, also KI-gestützt. Welche Variante ist für einen CMDB Chatbot zu bevorzugen?
\item [AH:] Ich denke im Hinblick auf die Betriebssicherheit, die wir herstellen bzw. einhalten sollten wir auf einen regelbasierten Bot setzen. Also wir entwickeln ganz klar das Regelwerk, wovon der Bot nicht abweichen darf. Sonst könnte ich mir vorstellen, dass es eine Art von Entwicklung gibt, die nicht in unserem Interesse ist. Da werden dann eventuell Aktionen durchgeführt, die nicht geplant waren, sondern die der Bot in dem Moment durch seine KI für richtig gehalten hat, was in dem Fall zu einer Katastrophe oder Störung führt.
\item [KW:] Und du denkst auch, dass die Mitarbeiter leicht die Sprache des Bots lernen?
\item [AH:] Ja, das glaube ich auf jeden Fall.
\item [KW:] Denkst du, dass Chatbots auch für andere Tätigkeiten im Rechenzentrum geeignet wären?
\item [AH:] Ja, das glaube ich. Die einfachen täglichen Routinetätigkeiten, Konfigurationstätigkeiten, wie die bereits erwähnten Filesystemerweiterungen und DNS-Einträge oder das Starten einer Wartung im Monitoring, vielleicht sogar das Aufsetzen eines neuen Server über einen Bot. Ich sehe da großes Potential. Man darf halt als Mensch, als Admin, nicht die Angst vor solcher Technologie entwickeln und fürchten, dass man ersetzt wird, sondern man muss es als sinnvolle Ergänzung der täglichen Arbeit sehen.
\item [KW:] Denkst du, dass andere Unternehmen, die schon eine CMDB einsetzen auch Verwendung für sowas haben könnten?
\item [AH:] Ich sehe da keine Beschränkung. Jeder der eine CMDB einsetzt, kann von so etwas nur profitieren.
\end{list}
