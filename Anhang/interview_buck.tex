\chapter{Experteninterview Buck}\label{Interview_Buck}

Kevin Weiss (\textbf{KW}) führte mit Konrad Buck(\textbf{KB}) ein Experteninterview  durch, um Anforderungen an einen CMDB Chatbot zu ermitteln. Herr Buck ist PR Manager für die i-doit CMDB.

\section*{CMDB im Rechenzentrum}

\begin{list}{X:}{\setlength{\labelsep}{5mm}}
\item[KW:] Was verstehen Sie denn unter einer CMDB? 
\item[KB:] Ja, also eine Configuration Management Database, die möglichst genial und dynamisch und clever alle mögliche Items aufnehmen und verwalten kann, um den IT-Betrieb zu gewährleisten, also einen sauberen, zeitsparenden, effektiven IT-Betrieb gewährleisten. Prozesse planen, Changes durchführen, später mal, aber zunächst einmal den IT-Betrieb organisiert sicherstellen.
\item[KW:] Ok, welche Funktionen finden Sie da besonders wichtig?
\item[KB:] Zunächst mal finde ich die UI total wichtig, also gibt es eine grafische Darstellung des Rechenzentrums oder des Netzwerks und kann man in die Räume herein zoomen. Also die UI halte ich für sehr wichtig, weil heute alles an Video und möglichst einfache bildliche Darstellung gewöhnt ist. Also diese ganze Listenanzeige etc. und diese ganzen Klickwüsten, die man häufig vorfindet, finde ich nicht so gut. Auch die Benutzerführung muss gut sein. Da finde ich es gut, wenn man ein System hat, das es einem erlaubt, möglichst viel zu machen, um sehr vieles von dem, was man selber in seinem Unternehmen machen möchte abzubilden und das dann zu einem Standard für seine CMDB Usage festzulegen. Also dass man das in einer entsprechenden Software tun kann, dass man also all diese möglichen Sachen aus dem Konfigurationsmanagement tun kann, also Wartungspläne, Notfallpläne, Arbeitsabläufe organisieren, eine Wissensdatenbank. Dann ganz wichtig, die Anbindung an weitere Systeme im IT-Servicemanagement, also Ticketing oder Inventarisierungssoftware. Schnittstellen muss sie bieten, damit man einen möglichst guten Überblick über das gesamte Geschehen hat. 
\end{list}

\section*{Chatbots allgemein}

\begin{list}{X:}{\setlength{\labelsep}{5mm}}
\item[KW:] Was verstehen Sie unter den Begriffen Chatbots und ChatOps?
\item[KB:] Chatbots sind Programme, die es einem erlauben, z. B. das Thema Ticketing zu automatisieren innerhalb einer CMDB Nutzung. Dass es eben nicht mehr ein Admin machen muss oder ein User mit dem Herstellersupport sprechen muss oder mit einem Mitarbeiter im Servicemanagement des Unternehmens, sondern dass vorgefertigte Standard Antworten und Fragen eben vorgefasst sind und dann von einem Bot beantwortet werden. Was die KI noch in Zukunft bringt mit intelligenten Bots, die selber sich das System angucken und sehen, wenn etwas nicht richtig läuft und eben auch schon einem Anrufer sagen können: Moment einmal, ich sehe, dass etwas nicht richtig läuft, das könnte die Ursache sein, ich prüfe das mal nach. Also das wäre so von den ganz einfachen Chatbots, die eben auch nur in einer Messaging Form schriftlich stattfinden, bis hin zu natürlichsprachlichen Bots, die dann mit dir reden.\\ Das ganze Thema ChatOps sagt mir noch nicht viel. Da muss ich gestehen, da habe ich mich noch nicht wirklich mit befasst. 
\item[KW:] Ja, das ist auch noch recht neu. In welchem Zusammenhang hatten Sie denn schon mit Chatbots zu tun?
\item[KB:] Also ich persönlich live im Grunde gar nicht. Ich habe nur diese ganzen Anwendungen bei der Telekom oder irgendwelchen Providern, die einen durch so eine Anfangsschleife leiten, bis man dann an einen richtigen Mitarbeiter kommt mit dem entsprechenden Anliegen.
\end{list}

\section*{Chatbots zur Bedienung der CMDB}

\begin{list}{X:}{\setlength{\labelsep}{5mm}}
\item[KW:] Welche Aufgaben kann ein CMDB Chatbot problemlos übernehmen und welche weniger gut?
\item[KB:] Er könnte problemlos alles das, was der Admin als Standard Anfragen, die immer wieder kommen und ihn immer wieder nerven übernehmen, indem man aus einer Wissensdatenbank die häufigsten Fragen zusammenfasst und intelligent in diese Chatbot Programme einbaut.  Da könnte man dann eben diese Verbindung herstellen und wenn man das sauber macht, würde der User das auch nutzen. Der kommt sich ja selber wahrscheinlich manchmal blöde vor, wenn er schon wieder dieselbe Frage stellt, weil er es sich einfach nicht merkt oder einfach kein Interesse daran hat, sich solche Banalitäten zu merken und dann muss der sich dann halt von einem Chatbot antworten lassen, also wiederkehrende Fragen.\\ Und dann, das ist aber wahrscheinlich schon eine Forderung an die Zukunft, nächstes Thema, Fragen bezüglich auflaufender Probleme, also predictive Aussagen machen zu können, zu dem, wie es im Netz aussieht. Eine Frage an so einen Chatbot könnte sein: Kann ich morgen mein neues Windows aufspielen, sind überall alle Systemanforderungen gegeben? Und dann müsste der Chatbot die hintergelagerten Systeme durchchecken möglichst schnell und sagen: Nein, bei dem Server ist noch nicht genug RAM da. Das wäre so die nächste Stufe aus meiner Sicht. Das könnten die heute im Grunde auch schon machen, ich wüsste aber keine Software, die das jetzt schon tut, keine CMDB. Der nächste Schritt wäre eine voll automatisierte UI, wo eben kein Bildschirm mehr gebraucht wird, sondern, ich möchte jetzt keine Produktnamen nennen, wo dann irgendwelche Sprachboxen stehen und dann könnte man sich als Admin oder User noch die Oberfläche ziehen von der CMDB oder von der Management Konsole, aber man kann auch über den Lautsprecher fragen: Was ist denn hier los? Dann erkennt der Bot anhand der Emotionalität, die in der Stimme steckt, das kann die KI ja zum Teil schon, was er da wohl meint und anhand des Profils dessen, der dann fragt und dann entsprechende Antworten liefern. Und das wäre dann wahrscheinlich schon das Ergebnis von ChatOps.\\ Nun gibt es bestimmt sehr viele Einzelaspekte innerhalb einer IT-Administration, die sehr tief in einer CMDB Nutzung drin steckt, ganz tolle Anwendungen gestrickt hat. Aus meiner Zeit bei synetics mit der CMDB i-doit kenne ich einige Anwender, die wirklich irre Sachen gemacht haben und die könnten vieles von dem, was sie da tun dann eben auch über Bots realisieren, vereinfachen auch. Da mit Sicherheit auch sowohl eine einfachere Handhabung machen für sich und ihre Kunden im Unternehmen und vielleicht auch eine gewisse Einsparung erzielen, wiederum an Zeit, um sich anderen Dingen widmen zu können und auch das Thema Akzeptanz bei den Endanwendern finde ich ganz wichtig, also wenn der Chatbot clever genug ist, könnte das die Akzeptanz von CMDBs oder überhaupt ITSM sozusagen quasi verbessern. 
\item[KW:] Was für Anforderungen haben Sie denn an einen CMDB Chatbot? Z. B. in Beziehung der Sicherheit oder Nachvollziehbarkeit.
\item[KB:] Also immer vorausgesetzt, dass ich kein dezidierter Anwender bin, sondern nur jemand, der über solche Software kommuniziert hat und daher die Dinge auch kennt. Also, ja ich meine man vertraut plötzlich einem Lautsprecher Informationen an als Unternehmen, genau so wie als Anwender. Der Anwender könnte jetzt zum Beispiel, wir waren ja eben bei diesem dümmsten anzunehmenden User, der zum 100. Mal fragt, wie etwas geht. Der könnte dann ja irgenwann einmal auch die Gefahr sehen, dass wenn er jetzt zum 101. Mal die gleiche Frage stellt, dass er dann wegen Blödheit entlassen wird. Also so was könnte ich mir vorstellen, dass Leute einem Menschen mehr vertrauen, als einem System, das eiskalt sagt: Das scheint mir nicht wirklich der richtige Mitarbeiter zu sein. Das bringt mich aber zu dem ganzen System Sicherheit, wobei das ja eigentlich egal ist. Die Sicherheit muss sowieso stimmen, ob da jetzt die Ausgabe ein Chatbot macht oder eine Textzeile, das ist glaube ich egal. Da verschiebt sich ja nur etwas in der Schnittstelle zwischen Mensch und Maschine. Insgesamt bietet dieser Bot dann auch noch Optimierungsmöglichkeiten in der Art und Weise in Anwendungsfeldern, die eine Tastaturkommunikation nicht hinkriegt. Da kann man Emotionalität nicht so gut erkennen. Das werden Chat Bots in Kürze können, aber die Sicherheit dahinter sollte immer gleich bleiben. Also ich finde z. B. auch das Thema interessant: Sicherheit nicht nur im Bezug auf Daten, sondern auch eben im Bezug auf Emotionalität des Users. Wenn wir das in Zukunft checken können mit KI-Chatbots, dann kommt da nämlich noch eine ganz andere Sicherheitsdimension dazu. Also heute kann keiner sehen, ob ich müde bin oder desinteressiert oder zerzaust bin, das wird KI in Zukunft sehr genau sehen können, wenn es das sehen darf. Das heißt, es kommt vor allem durch KI eine weitere Dimension der Sicherheitsaspekte hinzu.
\item[KW:] Rechnen Sie denn mit Problemen bei der Akzeptanz der Mitarbeiter?
\item[KB:] Ich denke, dass Chatbots es erst mal sehr schwer haben werden, sich von dem, was man in dem Telefonie Bereich kennt zu distanzieren. Also wenn die in der CMDB ähnlich daher kommen, wie bei der Telekom oder bei einem Versicherer oder der Krankenkasse, dann wird das schwer mit der Akzeptanz. Die müssen da irgendwelche cleveren Angänge finden, um anders zu klingen, anders zu reagieren und so weiter. 
\item[KW:] Ein Chatbot sollte immer eine gewisse Persönlichkeit haben, z. B. eine Figur aus Film und Fernsehen. Was würden Sie da für einen CMDB Chatbot sehen?
\item[KB:] Ich fände vielleicht Dschinni nicht schlecht, das ist aber eher eine spaßige Idee. Aber ich glaube bei Chatbots kommt es genau auf so etwas an, also wie schaffen wir es Vertrauen zu gewinnen und gleichzeitig auch die Professionalität auszustrahlen, die so ein Admin eben auch erwartet. Er will ja auf der einen Seite schon eben Bier trinken und Spaß haben, aber der will verdammt noch mal auch seinen Job machen. Das muss wirklich so eine Mischung aus Ernsthaftigkeit und Lockerheit sein. 
\item[KW:] Es gibt ja verschiedene Ansätze für einen Chatbot, einmal regelbasiert und einmal selbstlernend, also KI-gestützt. Was denken Sie wäre da für einen CMDB Chatbot besser geeignet?
\item[KB:] Da ich mich selber mit KI sehr beschäftige, würde ich sagen, besser gleich mit KI-Unterstützung. Die KI gibt es, die speech recognition ist mittlerweile recht gut, die kann auch Emotionalität ganz gut hinkriegen. Also da würde ich mich gar nicht mehr mit regelbasierten Dingen befassen, allerdings gibt es viele kleinere Unternehmen, die so etwas erst einmal machen müssen, weil sie das andere nicht bezahlen können.
\item[KW:] Haben Sie keine Bedenken bezüglich der Betriebssicherheit? Falls da etwas falsch interpretiert wird z. B..
\item[KB:] Da gehe ich von aus, dass die Leute die so etwas anbieten, bzw. für ein Unternehmen dann auch customizen, dass die das im Griff haben. Das würde ich voraussetzen, aber das ist vielleicht eine blauäugige Vorstellung von mir? Ich sehe da aber auch die Fallstricke. Aber die Möglichkeit, Fehlinterpretationen auszuschließen würde ich voraussetzen und ich glaube das tun die Anbieter von KI-unterstützten Systemen generell auch. 
\item[KW:] Sehen Sie denn Potenzial für andere Unternehmen als das IVZ, einen CMDB Chatbot zu verwenden?
\item[KB:] Ja, ich meine, wenn Sie da ein cooles Ding aufsetzen und an der Stelle freue ich mich, dass Sie das tun. Das war nämlich im Vorfeld gar nicht so sichtbar oder klar. Von daher finde ich das sehr spannend und ich kann mir in der Tat vorstellen, dass das jeder haben will. Erstens, weil es neu ist, zweitens, wenn es KI-gestützt ist einfach sehr arbeitserleichternd sein wird und wir reden im Moment viel zu viel über KI, das ist ein Stück weit auch klar, weil das noch relativ neu ist und wir haben zu wenig konkrete Anwendungen und jede, die da gut funktioniert und endlich auf den Markt kommt, ist gut. Da sehe ich große Chancen für. Ich glaube, wenn Sie das Thema Datensicherheit im Griff haben, dass da nichts falsch läuft, dann werden Sie da einen Blumentopf gewinnen und da fällt mir noch ein in dem Bereich: Es gibt so viele Möglichkeiten, sich zu vertippen und die kann eine Sprachsteuerung, wenn man so will, durch Plausibilitätsprüfung, die sie unbedingt machen muss, auch aus Sicherheitsgesichtspunkten, dann wunderbar abfangen. Also wenn Sie KI einsetzen, wird das System in diesem Bereich so wirkmächtig sein, dass es all diese Fehler dann auch mit abstellen kann und wird damit wahrscheinlich per Typos zustande kommende Fehler oder auch andere Anwendungsfehler einfach vermeiden. Das sehe ich im Übrigen auch noch als Entwicklungsprojekt bei vielen CMDBs, dass sie einfach diese eben angesprochene einfach vermeiden. Dass sie da so Plausibilitätsprüfungen oder andere Sachen einführen, wo dann gefragt wird: Willst du das jetzt wirklich durchführen?  
\end{list}
