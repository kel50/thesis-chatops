\chapter{Experteninterview Kirsten}\label{Interview_Kirsten}

Kevin Weiss (\textbf{KW}) führte mit Daniel Kirsten (\textbf{DK}) ein Experteninterview  durch, um Anforderungen an einen CMDB Chatbot zu ermitteln. Herr Kirsten ist Produktmanager für die i-doit CMDB.

\section*{CMDB im Rechenzentrum}

\begin{list}{X:}{\setlength{\labelsep}{5mm}}
\item[KW:] Was verstehst du unter einer CMDB?
\item[DK:] Eine CMDB, streng genommen nach ITIL steht für Configuration Management Database. Grundsätzlich muss man hier ein Bisschen differenzieren. Die Definition streng nach ITIL umfasst, dass die Elemente einer IT-Infrastruktur zentral erfasst werden und die Konfigurationen dieser Elemente und die Konfigurationen sind nicht 100-prozentig genau definiert, was da alles dazu gehört und es ist auch so, dass das Verständnis im Allgemeinen sehr stark sich unterscheidet, das können reine Asset Management Informationen sein, das können aber tatsächlich auch technische Konfigurationen sein. Nach ITIL spricht man heutzutage auch gar nicht mehr von einer reinen CMDB, sondern von einem sogenannten CMS, was wiederum ein Konstrukt ist, was auch nicht unbedingt eine zentrale Softwareplattform sein muss, sondern eine Vereinigung mehrerer CMDBen, die letztendlich unterschiedliche Datenquellen sein können. Was ich jetzt letztendlich unter einer CMDB verstehe, ist halt sehr unterschiedlich, weil ich in meiner Rolle als Produktmanager natürlich diese offizielle Definition kenne, die ITIL letztendlich vertritt oder herausgegeben hat. Grundsätzlich, das Gros der IT-Administratoren und Anwender gehen relativ verschwenderisch mit dem Begriff um und es wird halt in ganz vielen unterschiedlichen Ausprägungen genutzt. Es kann halt ein reines Asset Management sein, es kann halt auch eine sehr technische Datenbank sein. Wenn mich so jemand fragen würde, würde ich halt entsprechend sagen, es sollte idealerweise die Datenbank sein, die alle Komponenten einer IT-Infrastruktur vereint, dokumentiert und im Idealfall mindestens die technischen Grundkonfigurationen dieser Elemente aufnimmt.
\item[KW:] Wozu nutzt du die CMDB bei der täglichen Arbeit? Habt ihr so was auch selbst im Einsatz für die Komponenten?
\item[DK:] Wir selber nutzen natürlich unsere CMDB auch hier inhouse. Natürlich, um unsere inhouse-technischen Komponenten zu dokumentieren, dabei dient das bei uns hier intern tatsächlich als Nachschlagewerk für die tägliche, praktische Arbeit. Das heißt, wir haben hier keine automatisierten Prozesse oder Ähnliches, sonder wir machen eine manuelle Datenerfassung und erfassen hier die Dokumentation unserer internen und in der Cloud genutzten IT-Komponenten.
\item[KW:] Welche Funktionen findest du da besonders wichtig?
\item[DK:] Besonders wichtig ist für uns vor allen Dingen tatsächlich die Erfassung der IP-Adressen und auch Passwörter. Das ist so das, was hauptsächlich hier verwaltet wird. Wir haben eigentlich nahezu gar keine IT mehr inhouse, haben tatsächlich vor drei Wochen unsere letzten größeren physikalischen Server abgelöst. Das heißt wir dokumentieren sehr viel, welche Komponenten jetzt tatsächlich wo in der Cloud laufen und welche Services auch damit verknüpft sind.
\end{list}

\section*{Chatbots allgemein}

\begin{list}{X:}{\setlength{\labelsep}{5mm}}
\item[KW:] Was verstehst du unter Chatbots und ChatOps?
\item[DK:] Also der Begriff ChatOps ist mir vorher noch nicht untergekommen. Ich kenne natürlich den Begriff DevOps, weiß auch grundsätzlich, was damit gemeint ist. Ein Chatbot als Solches ist mir sehr geläufig. Ich hab tatsächlich lange Zeit, fast zehn Jahre lang Eggdrop TCL Scripte geschrieben für Eggdrop Bots, hab also quasi das Thema Chatbots sehr ausführlich behandelt, hab da einiges mit gemacht. Am Ende des Tages ist ein Chatbot immer eine Instanz, die letztendlich User Input nimmt und andere Datenquellen anzapft und den Input so normalisiert und runterstrippt, dass er in einem menschenlesbaren Format am Ende ausgegeben wird. Das Thema ChatOps habe ich jetzt in dem Zusammenhang mal kurz gegooglet, vorab zu den Interviews und so weit ich das verstanden habe, soll es ja letztendlich eine durch einen Chatbot unterstützter Prozess für DevOps sein.
\item[KW:] Genau. Welchen Nutzen und welche Vorteile siehst du in Chatbots? 
\item[DK:] Also grundsätzlich, wie gesagt, habe ich sehr sehr viel Erfahrung mit diesen Chatbots, man muss immer ein bisschen vorsichtig sein, dass man nicht zu viele Informationen versucht in so einen Chatbot hereinzusetzen. Grundsätzlich hat es immer dann den Vorteil, wenn man schnell Informationen benötigt und in Anführungszeichen zu faul ist, die passende Applikation dazu aufzumachen. Also ich kann mir vorstellen, wenn ich jetzt auf eine CMDB zugreife und ich möchte einfach sehr sehr schnell Informationen haben, dann könnte ich mir vorstellen, dass ich die gängisten Funktionen, die ich sonst in der CMDB nutze zur Verfügung stelle, um in der Lage zu sein, einfach sich diese 5 extra Klicks über einen Browser über eine Anmeldung etc. zu machen.
\end{list}

\section*{Chatbots zur Bedienung der CMDB}

\begin{list}{X:}{\setlength{\labelsep}{5mm}}
\item[KW:] Was denkst du welche Aufgaben ein CMDB Chatbot problemlos übernehmen kann und welche eher weniger gut?
\item[DK:] Ich könnte mir sehr gut vorstellen, dass gewisse häufig genutzte Funktionen von so einem Chatbot übernommen werden können, also wichtig ist immer aus meiner Sicht, dass zu komplexe Prozesse nicht unbedingt von so einem Bot übernommen werden, weil einfach der Output limitiert ist. Ich kann keine Grafik darstellen, ich kann also nur Textinformationen darstellen. Ich kann mir zum Beispiel sehr gut vorstellen, dass ich IP-Adressen von Systemen abfrage, dass ich zum Beispiel Standortdaten von Systemen abfrage oder Kontaktpersonen. Das sind so Dinge, wo ich sage: Ich will jetzt schnell Informationen über ein System haben oder vielleicht, wenn ich den kompletten Namen oder irgendetwas nicht weiß, so eine Minimalsuche habe, wo ich dann mit wildcards oder Ähnlichem versuche, aus der CMDB mir komplette Namen oder Ähnliches herauszusuchen. Was ich mir noch vorstellen könnte, wäre zum Beispiel, dass ich mir neue freie IP-Adressen reserviere, dass ich vielleicht sage: Okay, ich triggere \textit{getip}, gib den Netznamen an und bekomme die nächste freie IP-Adresse reserviert. Das sind Dinge, die ich mir vorstellen könnte. Was ich mir noch vorstellen könnte, wäre, dass ich als Administrator Änderungen am Logbuch sehr schnell erfasse. Dass ich zum Beispiel einen Patch eingespielt habe und ich habe jetzt auch wieder keine große Lust, auf das i-doit System zu connecten, sondern ich schreibe schnell \textit{addnotes "Gerade Patch eingespielt"}. Das sind Dinge, die ich mir gut vorstellen könnte, wobei hier, glaube ich, die Ausgabe von Informationen eher im Vordergrund steht, dass ich halt aktiv Dinge suche, bzw. über den Bot finde. Die Eingabe von Informationen ist wahrscheinlich eher im Hintergrund, da das von der Normalisierung her und von der Fehleranfälligkeit her nicht ganz so viel Sinn macht.
\item[KW:] Welche Anforderungen muss denn ein CMDB Chatbot erfüllen?
\item[DK:] Ja, das ist jetzt sehr allgemein gefragt.
\item[KW:] Z.b. in Richtung Sicherheit oder Nachvollziehbarkeit.
\item[DK:] Ein sehr schwieriges Thema bei so etwas ist natürlich immer das Rechtesystem. Ich muss natürlich gucken, dass der Chatbot nur die Informationen tatsächlich darstellt, die den Usern, denen der Chatbot zur Verfügung steht auch zugänglich sind. Wenn ich also mit einem sehr restrizierten Rechtesystem arbeite, wird es mit dem Einsatzszenario für den Chatbot eher schwierig, weil Leute auf die Idee kommen könnten, das Ding zu missbrauchen, um Informationen abzurufen, auf die sie keinen Zugriff haben sollten. Natürlich muss ich halt gucken, dass ich so einen Chatbot nur im internen Netz betreibe, um halt nicht meine CMDB ins Internet zu exposen. Da müsste man halt entsprechend vorsichtig sein. Ich sage mal, wenn ich jetzt so einen Chatbot bauen würde technologisch, würde ich halt sagen, Slack ist der absolute Standard heutzutage im Unternehmen. Das wäre halt das erste, worauf ich das Ding konzenrieren würde. Grundsätzlich würde ich aber gucken und da spricht der Produktmanager in mir, dass ich so einen Chatbot plattform-agnostisch schreibe, d.h. ich würde ihn so schreiben, dass er ins Backend auf die CMDB zugreifen kann und ich nachher in der Lage bin, verschiedene Chatsysteme daran anzubinden, sei es jetzt Slack oder andere.
\item[KW:] Rechnest du denn mit Problemen beim Einsatz von Chatbots, auch in Bezug auf die Akzeptanz der Mitarbeiter?
\item[DK:] Akzeptanz der Mitarbeiter in sofern nicht, weil mit solchen Dingen ist es immer so: Es muss halt einer nutzen und meistens werden diese Bots halt in öffentlichen Channels benutzt und das ist dann eher so was Virales. Einer fängt halt an, das zu nutzen, der Andere findet das cool, sieht das in nem Channel und dann verselbständigt sich das. Allerdings bekommt man auch null Feedback, ob da was fehlt oder was nicht passt. Deswegen sehe ich jetzt keine Gefahr oder Ähnliches in dem Sinne, weil wenn es nicht gut funktioniert, wird es einfach nicht genutzt. Dann ist auf gut deutsch gesagt für die Katz, so ne Entwicklung. Ich glaube aber, dass diese kleinen nützlichen Funktionen schon sehr gut akzeptiert werden.
\item[KW:] Es gibt ja verschiedene Bot Frameworks, die in Python, Ruby oder Coffeescript geschrieben sind. Was denkst du ist für die Weiterentwicklung da am Besten?
\item[DK:] Ich kenne mich da jetzt nicht so wahnsinnig aus in den aktuellen Technologien, also wenn ich jetzt Python höre, das ist natürlich eine sehr mächtige Scriptsprache, die nach wie vor sehr weit verbreitet ist. Da hätte ich auf jeden Fall wenig Bauchschmerzen, dass das kompatibel ist. Coffeescript sagt mir jetzt gar nicht so viel. Also neuere Technologien, die noch nicht ganz gesettled sind, da wäre ich vorsichtig an der Stelle.
\item[KW:] Ein Chatbot sollte immer eine gewisse Persönlichkeit haben. Was kannst du dir da für einen CMDB Chatbot vorstellen?
\item[DK:] Ich finde, das Ding zu personifizieren an der Stelle würde ich als nicht richtig erachtet, weil die Leute, die damit arbeiten sind keine Endkunden, sondern sind Techniker, die schnörkellos Informationen haben wollen oder eingeben wollen und nicht unbedingt mit Hallo und Guten Tag begrüßt werden wollen. Also ich würde, wenn überhaupt diesem Ding eine Roboter-Personalität geben. Dass das Ding, wenn überhaupt sich das personifiziert darstellt, sich auch als Roboter darstellt.  
\item[KW:] Es gibt ja verschiedene Ansätze für Chatbots, also regelbasiert oder selbstlernend, bzw. KI-gestützt. Was denkst du ist da für einen CMDB Chatbot die bessere Wahl?
\item[DK:] Also ich denke primär würde ich das Ganze regelbasiert aufbauen. KI-gesteuert ist bestimmt interessant, nur glaube ich ist es schon sehr schwierig, den Bot darauf zu trainieren, weil der Anwender ja auch Feedback geben muss, ob das was er präsentiert bekommt überhaupt das ist, was er gesucht hat. Ich weiß nicht, ob diese Feedback Schleifen den Anwender nicht zu sehr frustrieren würden, denn er wird halt vortrainiert aber wenn ich jetzt einem Endanwender so etwas präsentiere und er muss quasi 3 mal erst mal sagen, das ist nicht das, was ich gesucht habe, oder zu viel oder zu wenig, dann ist da aus Anwendersicht schon irgendwie eine gewisse, also fehlt eine gewisse Ernsthaftigkeit und führt halt auch letztendlich zu Frustrationen auf Anwenderseite. Ich finde das interessant, technologisch. Das kann auch zu sehr interessanten Ergebnissen führen, ich würde mich aber primär erst einmal auf kleine, regelbasierte Aufgaben konzentrieren.  
\item[KW:] Siehst du auch Potential für andere Unternehmen, so einen CMDB Chatbot zu verwenden?
\item[DK:] Ohne Frage. Ich glaube, wenn so etwas existiert, dann hat das ganz großes Potential, dass das häufig eingesetzt wird. Man sieht anhand von Slack, dass die Unternehmenskommunikation immer stärker in Richtung Instant Messenger geht, das hat viel damit zu tun, dass die Leute privat auch mit der WhatsApp-Nutzung auch sehr vetraut sind und mit dem Medium mittlerweile professionell umgehen können und das ist ein absolut zukunftsträchtiges Medium und da Schnittstelle zu haben ist mehr als sinnvoll.
\item[KW:] Ok, dann wären wir auch schon durch.
\item[DK:] Ich darf noch eine kurze Sache ergänzen: Ich wäre vorsichtig mit Funktionen, die automatisch Dinge darstellen. Es gibt vielfach Chatbots, die automatisch immer Neuigkeiten, Dinge, die passieren in irgendwelche Channel rein posten. Das wird ganz häufig genutzt von vielen Leuten und letztendlich guckt aber dann keiner hin, weil halt Informationen immer wieder verloren gehen. Ich würde es tatsächlich so designen, dass das auf Nachfrage reagiert und nicht proaktiv irgendwelche Informationen spammt.
\end{list}
