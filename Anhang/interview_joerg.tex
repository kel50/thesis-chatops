\chapter{Experteninterview Middendorf}\label{Interview_Middendorf}

Kevin Weiss (\textbf{KW}) führte mit Jörg Middendorf (\textbf{JM}) ein Experteninterview  durch, um Anforderungen an einen CMDB Chatbot zu ermitteln. Herr Middendorf ist CMDB Verantwortlicher beim IVZ.

\section*{CMDB im Rechenzentrum}

\begin{list}{X:}{\setlength{\labelsep}{5mm}}
\item[KW:] Was verstehst du unter einer CMDB? 
\item[JM:] Eine CMDB ist eine Configuration Management Datenbank, in der alle Configuration Items, bzw. alle Gerätschaften eines Unternehmens dokumentiert sind und in Relation gesetzt werden, also wie sie zusammenhängen oder korrelieren.
\item[KW:] Und wofür wird die CMDB eingesetzt?
\item[JM:] Bei uns im Rechenzentrum wird die CMDB eingesetzt, um alles was wir betreiben, wie Server, Lizenzen und Aufstellungsorte zu dokumentieren und für Supportfälle, zur Dokumentation und für Planungstätigkeiten zu nutzen.
\item[KW:] Wozu nutzt du die CMDB bei der täglichen Arbeit?
\item[JM:] Zur Dokumentation neu erstellter Maschinen. Um nachzugucken, wo ein Server steht, z. B. wenn ein Supportfall da ist. Um Reports für Planungen zu erzeugen und zu erfassen, wie viele Server an einem Standort überhaupt sind.\\ Wir haben jetzt ein Projekt, bei dem Rechenzentren umgezogen werden, da müssen wir natürlich wissen, um wie viele Komponenten es sich handelt. Außerdem zur Dokumentation von IP-Netzen, wenn ich eine neue IP-Adresse vergeben oder gucken möchte, wofür sie genutzt wird. 
\item[KW:] Und welche Funktionen findest du denn besonders wichtig, auch im Hinblick auf die Zusammenarbeit mit anderen Tools?
\item[JM:] Wir haben eine Schnittstelle zum Monitoring System, also man hat die Möglichkeit, Daten aus dem Monitoring abzufragen und auch aus der CMDB geplante Daten ins Monitoring zu transferieren, ohne sich im Monitoring anmelden zu müssen. Das ist eine gängiger Schnittstelle und wichtige Funktion. Ansonsten ist die Reporting-Funktion sehr wichtig im täglichen Doing.
\end{list}

\section*{Chatbots allgemein}

\begin{list}{X:}{\setlength{\labelsep}{5mm}}
\item[KW:] Was verstehst du unter den Begriffen ChatOps und Chatbots?
\item[JM:] Unter ChatOps verstehe ich eine Art Toolchain, die Entwickler und Administratoren zusammenschaltet und Anforderungen definiert, die automatisch getestet werden und dann in der Produktion eingesetzt werden können.
\item[KW:] In welchem Zusammenhang hattest du schon mit Chatbots zu tun?
\item[JM:] Ein Chatbot ist ja eigentlich eine Software, die auf Eingaben, also Text oder Sprache, reagiert. Da hat man mit den gängigen Sprachassistenten, die heutzutage im Umlauf sind Kontakt, sei es Siri, Alexa oder was es da alles gibt. Im eigentlichen textbasierten Umfeld gibt es sowas, wenn man mit Supportanbietern kommuniziert. Da sind die dafür da, um erst mal das Anliegen aufzunehmen und zu klassifizieren. Da gibt es verschiedene Varianten, es gibt Chatbots, die reagieren auf Sätze, es gibt aber auch welche, die reagieren auf Befehle. Also mit solchen Dingen hatte ich schon Kontakt.
\item[KW:] Welche Vorteile siehst du allgemein im Einsatz von Chatbots?
\item[JM:] Im Rechenzentrumsbereich sehe ich viele Vorteile. Für Standardaufgaben müssen dann nicht mehr die Mitarbeiter herumlaufen und andere befragen, z. B. wo ein Server steht, wo ein Gerät sich befindet, welche Software darauf läuft und welche IP-Adresse es hat. Da hat man bisher andere gefragt oder Mails geschrieben und auf eine Antwort gewartet und mit einem Bot entfällt das. Die Leute im Rechenzentrum, die alle Experten sind, können sich dann um die eigentliche Aufgabe kümmern und müssen sich nicht mit diesen Standardaussagen auseinandersetzen.
\end{list}

\section*{Chatbots zur Bedienung der CMDB}

\begin{list}{X:}{\setlength{\labelsep}{5mm}}
\item[KW:] Was denkst du, welche Aufgaben ein CMDB Chatbot gut übernehmen kann und welche weniger gut?
\item[JM:] Ein Chatbot, der mit der CMDB zusammenarbeitet bietet einige Vorteile. Ich kann in einem Chattool, das ich auf dem Handy oder Rechner habe abfragen, wo ein Server steht. Also angenommen ich gehe jetzt ins Rechenzentrum und möchte wissen, in welchem Rack der steht. Dann kann ich nachschauen und weiß schon auf dem Weg, wo ich gucken muss. Ich kann alles, was für diese Item konfiguriert ist abfragen, ich kann gucken welche Software darauf läuft und welche IP-Adresse das Gerät hat. Zu welchem Subnetz es gehört, welches VLAN konfiguriert ist, wie viel Festplattenplatz verfügbar ist und wie stark der RAM ausgelastet ist. Das kann ich alles in einem herkömmlichen Chat Tool abfragen, ohne die Software öffnen zu müssen, die dann vielleicht auf Mobilgeräten nicht richtig angezeigt wird. Ich kriege die ganzen Anforderungen dann in meinem Tool, das ich dabei habe.
\item[KW:] Ok, also welche Arbeitsschritte würde ein Chatbot vereinfachen?
\item[JM:] Z. B. wenn ich im Rechenzentrum einen Server einbaue, kann ich dem Chatbot sagen, dass der Server mit dem Namen \textit{Sowieso} im Rack \textit{XY} an Platz \textit{123} steht und der dokumentiert das dann für mich in der CMDB oder legt mir den Server grundlegend an mit Name und IP Adresse, damit ich die grundlegenden Informationen vor Ort eintragen kann. Im Umkehrfall baue ich einen Server aus und setze ihn per Chatbot in der CMDB auf einen anderen Status, also \textit{außer Betrieb} oder \textit{verschrootet}. Das kann man direkt vor Ort vereinfachen.
\item[KW:] Welche Anforderungen hast du an einen Chatbot der mit der CMDB interagiert?
\item[JM:] Zum Einen sollte der Chatbot keine Anfragen annehmen von Personen, die in der CMDB keine Berechtigungen haben. Entweder sagt man dem Chatbot, dass er nur Anfragen von bestimmten Personen entgegennehmen darf oder, dass er nur in einem bestimmten Gruppe aktiv ist, in den nur berechtigte Personen eingeladen werden, die dann diese Tätigkeiten vornehmen können. Also zumindest für das Schreiben in die CMDB, für das Lesen kann man vielleicht eine andere Regelung finden, z. B. jeder der im Rechenzentrum ist darf Abfragen machen: \textit{Wo ist der Server?}, \textit{Wer ist dafür zuständig?}, usw. . Die Anforderung an den Bot wäre dann noch, dass er die gängigen Methoden der CMDB nutzt, also APIs oder die entsprechenden Command Line Tools, die der CMDB Hersteller anbietet und nicht direkt in die Datenbank schreibt.
\item[KW:] Bei der Benutzung des Bots wird der Bot User als Ersteller vermerkt. Soll auch berücksichtigt werden, wer mit dem Bot interagiert hat?
\item[JM:] Ja, in einem Log File oder im Chatverlauf sollte nachvollziehbar sein, wer welche Änderung vorgenommen hat und dies sollte auch dauerhaft abgelegt werden. Im Optimalfall würde der Chatbot bei einer Änderung direkt in der CMDB vermerken, von welchem Nutzer die Änderung kommt. Der Chatbot dient dann quasi nur als ausführende Kraft
\item[KW:] Rechnest du mit Problemen beim Einsatz von Chatbots?
\item[JM:] Eigentlich sehe ich da nur Vorteile. Wie ich bereits am Anfang erzählt habe, ist es aktuell so, dass die Mitarbeiter in die Büros gehen und fragen: \textit{Was ist das für ein Server? Wo steht der?}. Sie stellen immer die gleichen Fragen und diese Fragen können sie dann über einen Chatbot selbst beantworten. Also eigentlich sehe ich da keine großen Probleme. Das Problem ist halt, wenn die Chat Software irgendwo außerhalb des Unternehmens steht und man unternehmenskritische Dinge darin postet. Da muss natürlich sichergestellt sein, dass die entsprechende Regularien haben, dass diese Daten nicht irgendwann im Netz auftauchen oder weiterverarbeitet werden.
\item[KW:] Der ChatOps lebt von Transparenz, also es findet alles in einem Raum statt und man kann sehen, was die anderen machen. Denkst du es gibt bei den Mitarbeitern Probleme bezüglich der Akzeptanz?
\item[JM:] Nein, eigentlich nicht. Das einzige was ich mir vorstellen könnte ist, dass es als Arbeitskontrolle wahrgenommen wird. Es gibt vielleicht einen Mitarbeiter, der am Tag 10 Änderungen macht und der andere nur eine. Da könnte man sagen, dass er zehn mal so viel arbeitet. Sowas darf man natürlich nicht auswerten. Aber ich sehe in der Transparenz auch viele Vorteile. Wenn jemand eine Änderung vornimmt können alle anderen sich das da abholen und sehen direkt, dass sich was in der Umgebung geändert hat. 
\item[KW:] Siehst du denn auch Vorteile darin dass andere dann vielleicht etwas lernen können, also wie man einen Arbeitsvorgang durchführt?
\item[JM:] Ja, erstens das und sie können sich selbst über Sachen informieren, die passiert sind. Nicht wie früher, dass man eine Änderung vornimmt und dann eine E-Mail schreibt, die keiner liest. So kann man in den Chat Raum reingehen und z. B., wenn man aus dem Urlaub kommt nachschauen, was in den letzten beiden Wochen pasiert ist.
\item[KW:] Es gibt verschiedene Bot Frameworks, die in Ruby, Python oder Coffeescript geschrieben sind. Was denkst du wäre das Sinnvollste in Betrachtung der Weiterentwicklung?
\item[JM:] Also ich würde Errbot präferieren, weil er in Python geschrieben ist und wir hier Python Know-How vor Ort haben und auch die meisten Leute im Rechenzentrum Python Code lesen können. Sofern der alle Anfoderungen im Laufe der Zeit erfüllen kann, würde ich den bevorzugen.
\item[KW:] Bei einem Chatbot ist es vorgesehen, dass er eine gewissse Persönlichkeit hat. Also z. B. eine  Figur aus Film und Fernsehen. Welche Figur würdest du mit einem CMDB Chatbot verbinden?
\item[JM:] Ich bin da relativ neutral eingestellt. Ich finde es ganz lustig, wenn die eher so einen Comedy Charakter haben. Also wenn ich die gleiche Abfrage zwei mal stelle, dass er dann irgendwie einen süffisanten Spruch von sich gibt. Das fände ich ganz lustig. Eine eigene Person würde mir da nicht einfallen.
\item[KW:] Ok, also er sollte dann vielleich eher einen frechen Charakter haben, bei Falscheingaben usw.?
\item[JM:] Ja, da muss man gucken, dass das für alle Kulturen, die hier im Unternehmen vertreten sind auch lustig wirkt. Es kann ja schon mal sein, wenn man eine andere Kultur hat, dass die Sachen, die andere lustig finden dann verletzend finden. Da muss man natülich gucken. Ich finde bei so einem Bot eher besser, dass er nüchtern reagiert, also wenn ich ihm einen Befehl gebe, soll er genau das tun, was ich sage und nichts interpretieren. Das fände ich wichtig. Wenn der Bot anfängt zu interpretieren, kann das Ergebnis ganz ander aussehen, als das, was man beabsichtigt hat. Man sollte natürlich drauf achten, dass er Bot Schritte automatisiert und verlässlich bearbeitet. Nicht dass er nachher irgendwelche Dienste im Rechenzentrum abschaltet oder herunterfährt, nur weil er es anders interpretiert hat, als der Mensch es meint.
\item[KW:] Ok, also denkst du, man sollte bei kritischen Aktionen auch nochmal eine Nachfrage machen?
\item[JM:] Ja, das würde Sinn machen. Ich fände es auch gut, wenn der Bot bei einer Eingabe in die CMDB auch Auswahlmöglichkeiten zur Verfügung stellt, wenn man Sachen eingeben will vielleicht schon in einer Form oder Schreibweise vorhanden sind. Wir haben jetzt so ein Beispiel beim Hersteller. Da gibt es \textit{Hewlett Packard}, \textit{HP}, dann gibt es \textit{HPE}, also ganz viele Schreibweisen für die gleiche Firma. Da sollte er dann die Übersetzung machen, wenn man sich verschrieben hat und den korrekten Namen vorschlagen. Das Ganze sollte dann interaktiv sein.
\item[KW:] Es gibt verschiedene Arten von Chatbots. Regelbasiert oder KI-basiert, also selbstlernend. Du hast jetzt die Zuverlässigkeit, die der Bot erfüllen soll schon erwähnt. Denkst du denn auch, dass er eher regelbasiert sein sollte?
\item[JM:] Also zumindest im Rechenzentrumseinsatz bin ich ein Fan von regelbasierten Sachen. Da will man einfach wenn man was dokumentieren oder abfragen möchte, dass er auch nur das macht und nicht noch andere Sachen, die er dazu gelernt hat. Es kann aber auch durchaus sein, dass eine künstliche Intelligenz für andere Anwendungsfälle Sinn machen würde. Also z. B., wenn ein neuer Mitarbeiter ins Unternehmen kommt und fragt, wo die Dokumente sind. Da könnte man dann was Anderes nehmen, aber hier für den Rechenzentrumseinsatz würde ich auf jeden Fall eine regelbasierte Lösung bevorzugen.
\item[KW:] Denkst du, dass die Mitarbeiter einfach die Sprache des Bots lernen können?
\item[JM:] Ich denke nicht, dass das ein Problem ist. Die meisten sind hier auf Konsolen unterwegs und sind da recht firm drin. Wenn der Bot in Betrieb ist kann man ja auch eine Einführungsveranstaltung machen, um zu zeigen, welche Sachen der kann. Oder man kann den Bot fragen: \textit{Welche Befehle kennst du denn?}. Der sollte halt so eine Hilfe ausgeben oder ein paar Beispiele.
\item[KW:] Siehst du auch Potential für andere Rechenzentrumstätigkeiten?
\item[JM:] Ja, also alles, was automatisiert werden kann oder eine API hat, kann man damit wunderbar bearbeiten. Ich habe z. B. schon einmal einen Fall gesehen, bei dem virtuelle Maschinen provisioniert werden. Man sagt dem Bot, dass man eine Maschine braucht, die in gewisser Weise ausgeprägt ist und im Hintergrund laufen die entsprechenden Skripte und der Bot teilt einem dann mit, wann die Maschine fertig erstellt ist und wie die Daten sind, um darauf zu kommen. So was könnte ich mir sehr gut vorstellen. Es gibt auch andere Möglichkeiten, sich an Puppet z. B. anzudocken und da Konfigurationen zu erstellen und abzufragen, DNS Einträge zu machen, abzufragen, zu ändern. Da gibt es unzählige Möglichkeiten, die man benutzen kann. In der Vergangenheit war es halt schwierig, weil die meisten Softwarelösungen keine APIs hatten, da musste man in die Datenbanken schreiben und mittlerweile bietet fast jede Software auch APIs an und darüber kann man sehr schön mit einem Bot Aktionen auslösen.
\item[KW:] Siehst du auch für andere Unternehmen, die vielleicht schon eine CMDB einsetzen Möglichkeiten, ChatOps einzuführen?
\item[JM:] Ja, also alles, was automatisierbar ist, ist ja im Interesse von Rechenzentrumsbetreibern. Wir wollen nicht alle Aktionen zehn mal machen, es gibt ja auch nicht unbegrenzt Personal für solche Dinge. Alles was automatisierbar und machbar ist, ist auch für andere Unternehmen sehr sinnvoll und man nimmt mit so einem Chatbot natürlich die ganze Komplexität raus, ich sage dem, dass ich eine virtuelle Maschine brauche und das, was im Hintergrund passiert, kann ja sehr komplex sein und für den Anwender, der diese Anforderung erstellt, kann das trivial wirken. So einen komplexen Vorgang zu vereinfachen ist natürlich ein schöner Nebeneffekt.
\end{list}

