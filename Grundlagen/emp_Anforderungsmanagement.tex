\section{Anforderungsmanagement}
% \footcite[Vgl.][o. S.]{Grande_2014_Anforderungsmanagement}


% Ziel des Anforderungsmanagements bzw. Requirements Engineerings nach Pohl und Rupp ist die Erfassung und Dokumentation von Anforderungen.\footcite[Vgl.][S. 3]{Pohl_2015_Requirements}
%
% Die Haupttätigkeiten sind die Ermittlung, Dokumentation, Prüfung und Verwaltung von Anforderungen, wobei die Verwaltung mit den drei vorigen Tätigkeiten einher geht. \footcite[Vgl.][S. 4 f.]{Pohl_2015_Requirements}
% %
% \subsection{Ermittlung}
% Bei der Ermittlung der Anforderungen werden verschiedene Quellen, wie z. B. Stakeholder, Dokumente oder aktive Systeme genutzt um Anforderungen zu detektieren und zu verfeinern.\footcite[Vgl.][S. 21]{Pohl_2015_Requirements}\\
% Das Interview ist eine mögliche Befragungstechnik, bei der ein Stakeholder (Interviewpartner) vom Requirements Engineer (Interviewer) vorgegebene Fragen gestellt bekommt. Die Antworten werden protokolliert. Etwaige Rückfragen können sofort im Gesprächsverlauf geklärt werden. Durch geschickt gestellte Fragen können auch unterbewusste Anforderungen ermittelt werden. Der Verlauf des Gesprächs kann individuell angepasst werden, es wird gezielt nachgefragt und auf einzelne Themen eingegangen, um das behandelte Thema möglichst komplett abzubilden.\footcite[Vgl.][S. 28]{Pohl_2015_Requirements}
%
% \subsection{Dokumentation}
% Die erarbeiteten Anforderungen werden in natürlicher Sprache oder in dafür vorgesehenen Modellen dokumentiert, wobei entsprechende Techniken zum Einsatz kommen. \footcite[Vgl.][S. 4]{Pohl_2015_Requirements}\\
% Die im Requirements Engineering anfallenden Tätigkeiten müssen geeignet dokumentiert werden. Dazu zählt z. B. auch die Erstellung von Interviewprotokollen. Als Dokumentation gilt jede Form der formalen Darstellung, von Beschreibung in natürlicher über strukturierten Text bis hin zu formalen Techniken wie Diagrammen. Die Dokumentationsform sollte der zugrundeliegenden Aktivität angemessen sein und diese sinnvoll wiedergeben. \footcite[Vgl.][S. 35 ff.]{Pohl_2015_Requirements}

% \subsection{Prüfung}
% % Um die Qualität der Anforderungen gewährleisten zu können, muss diese frühzeitig geprüft werden.\footcite[Vgl.][S. 4]{Pohl_2015_Requirements}
% Zu den Qualitätskriterien zählen u.A. Korrektheit und Abgestimmtheit. Die Methoden der Qualitätskontrolle können dabei sowohl für einzelne Anforderungen, als auch für Anforderungsdokumente eingesetzt werden.
% Bei der Überprüfung der Anforderungen sollen Fehler wie \glqq{}Mehrdeutigkeit, Unvollständigkeit und Widersprüche\grqq\footcite[][S. 95]{Pohl_2015_Requirements} aufgedeckt werden.\footcite[Vgl.][S. 95]{Pohl_2015_Requirements}

\subsection{Verwaltung}
Die Verwaltung der Anforderungen erfolgt parallel zu den vorher genannten Tätigkeiten und beinhaltet die Strukturierung und Aufbereitung für verschiedene Rollen.\footcite[Vgl.][S. 5]{Pohl_2015_Requirements}\\
Außerdem werden Verfolgbarkeit, Versionierung, Priorisierung und die Verwaltung von Änderungen unter Beibehaltung der Konsistenz sichergestellt. Auch hier reicht der Fokus von Einzelanforderungen bis hin zu Anforderungskatalogen.
Die Anforderungen werden mit den Informationen Identifikator, Name, Autor und Quelle versehen. Die Priorisierung der Anforderungen kann z. B. durch den Auftraggeber oder Faktoren wie die Dringlichkeit der Umsetzung geschehen.\footcite[Vgl.][S. 126]{Pohl_2015_Requirements}
