\newcommand*{\quelle}{%
  ~\\\smallskip\footnotesize Quelle: \cite
%  \footnotesize Quelle: \cite
%  \footcites
}

\newcommand*{\quelleeigen}{%
  ~\\\smallskip\footnotesize Quelle: Eigene Abbildung nach: \cites
%  \footnotesize Quelle: Eigene Abbildung nach: \cites
%  \footnotesize (Quelle: Eigene Abbildung nach: \cite[#1]{#2})
}

\newcommand*{\eigen}{%
  ~\\\smallskip\footnotesize Quelle: Eigene Darstellung
  %\footnotesize Quelle: Eigene Darstellung
}




\newcommand{\HRule}{\rule{\linewidth}{0.5mm}} %Für Titelseite
\newcommand{\hsp}{\hspace{20pt}} %Für Chapter Überschriften
\newcommand{\note}[2][]{\todo[color={red!100!green!33}, #1]{#2}}
\newcommand{\question}[2][]{\todo[color=green, #1]{#2}}
\newcommand{\dotoday}[2][]{\todo[color=red, #1]{#2}}
\newcommand{\todoin}[2][]{\todo[inline, #1] {#2}}
\newcommand{\nicetohave}[2][]{\todo[color=blue!30, #1]{#2}}

% \presetkeys%
%     {todonotes}%
%     {inline}{}
%

\immediate\write18{texcount -inc -incbib -sum -total -nosub 01_Einleitung.tex 02_Grundlagen.tex 03_Hauptkapitel.tex 04_Betriebswirtschaftliche_Betrachtung.tex 05_Fazit.tex > /tmp/wordcount.tex}
\newcommand\wordcount{
\verbatiminput{/tmp/wordcount.tex}}
