\usepackage[T1]{fontenc} % Westeuropäische Codierung
\usepackage[utf8]{inputenc} % Umlaute
\usepackage{textcomp} % Verschiedene Textsymbole
\usepackage[ngerman]{babel} % Sprache des Dokuments. Beeinflusst Überschriften, Datumsformat, etc.
\usepackage[pdftex]{graphicx} % Bilder einbinden
\usepackage[table]{xcolor} % Eigene Farben definieren
\usepackage{fancyhdr} % Anpassbare Header und Footer
\usepackage[babel,german=quotes]{csquotes}
\usepackage{geometry} % Zum Anpassen der Seitenränder
\usepackage{blindtext} % Zum Erzeugen von Blindtext
\usepackage{setspace} % Um die Abstände von Überschriften zum Text anzupassen
\usepackage{libertine} % fancy Schriftart, die small capitals unterstützt
% \usepackage{helvet}
\usepackage{hyperref} % Verlinkungen und PDF Optionen
\usepackage{titlesec} % Überschriftenformat anpassen
\usepackage{booktabs} % rules für Tabellen
\usepackage[printonlyused]{acronym} % Abkürzungen, es werden nur die angezeigt, die benutzt werden
\usepackage{tabularx} % Tabellen über die gesamte Breite
\usepackage{caption} % Zur Anpassung von Titeln von Gleitobjekten
\usepackage{diagbox} % Überschrift mit Schrägstrich in Tabellenkopfzeile
\usepackage{float} % Einige Zusatzoptionen für float Objekte
\usepackage[backend=biber, style=authoryear-icomp, mergedate=false, dashed=false, maxbibnames=99]{biblatex} % Bibliografie, möglichst alle Autoren anzeigen
\usepackage[babel,german=quotes]{csquotes} % Zitationsoptionen
\let\counterwithout\relax
\let\counterwithin\relax
\usepackage{chngcntr}  % für fortlaufende Fußnotennummerierung
\usepackage[colorinlistoftodos]{todonotes} % ToDos
\usepackage{moreverb} % Datei Ein- und Ausgabe
\usepackage{listings} % Listings
\usepackage{blindtext}
\usepackage[normalem]{ulem} % Strike Through Text
\usepackage{rotating}
\usepackage{multirow}
\usepackage{appendix}
\usepackage[bottom]{footmisc}
