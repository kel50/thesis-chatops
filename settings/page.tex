\definecolor{gray75}{gray}{0.75}
\definecolor{fom}{HTML}{23a092}
% \font\myfont=libertine at 12pt
% \definefont[myfont][name:libertine at 12pt]

\renewcommand*\familydefault{\sfdefault}
% \renewcommand*\familydefault{\rmdefault}
\newcommand{\mynormal}{\fontsize{12}{12}\selectfont}
\newcommand{\mylarge}{\fontsize{14}{14}\selectfont}
\newcommand{\MyLarge}{\fontsize{17}{17}\selectfont}
\newcommand{\MYLARGE}{\fontsize{20}{20}\selectfont}

\newcommand{\fomhead}{
	\fancyhead{}
	\fancyfoot{} %alle Kopf- und Fußzeilenfelder bereinigen
	\fancyhead[R]{\mynormal Seite \thepage}
  \fancyhead[L]{\includegraphics[width=1cm]{Anhang/fom.jpg}}

	\renewcommand{\headrulewidth}{0.2pt} %obere Trennlinie
%	\fancyfoot[L]{\mynormal\autorkurz}
	\fancyfoot[L]{}
	\fancyfoot[C]{}
	\fancyfoot[R]{}
%	\renewcommand{\footrulewidth}{0.2pt} %untere Trennlinie
	\patchcmd{\headrule}{\hrule}{\color{gray75}\hrule}{}{}
%	\patchcmd{\footrule}{\hrule}{\color{gray75}\hrule}{}{}
}

\fancypagestyle{plain}{\fomhead}% Für Kapitelseiten, Inhaltsverzeichnis, etc nutzen
\fomhead% Für "`normale"' Seiten setzen

\setlength{\headheight}{33pt}
\onehalfspacing

% \geometry{a4paper,left=3.4cm,right=3.4cm, top=3cm, bottom=3cm}
\geometry{a4paper,left=3cm,right=3cm, top=3cm, bottom=3cm}

\titleformat{\chapter}[display] {\normalfont\huge\bfseries}{\textcolor{fom}{\chaptertitlename\ \thechapter}}{20pt}{\Huge\textcolor{fom}}
% \titleformat{\chapter}[hang]{\vspace{5ex} \LARGE\bfseries\scshape}{\textcolor{fom}{Kapitel \thechapter}\hsp}{20pt}{\LARGE\bfseries\textcolor{fom}}
% \titleformat{\chapter}[hang]{\vspace{5ex} \LARGE\bfseries\scshape}{\textcolor{fom}\thechapter\hsp\textcolor{gray75}{|}\hsp}{20pt}{\LARGE\bfseries\textcolor{fom}}
% \titleformat{\chapter}[display] {\normalfont\huge\bfseries}{\chaptertitlename\ \thechapter}{20pt}{\Huge}  % Default Titleformat

% \titlespacing{\chapter}{0mm}{-4em}{2.0em}
% \titlespacing{\section}{0mm}{-0.5em}{-0.3em}
% \titlespacing{\subsection}{0mm}{-0.5em}{-0.3em}
% \titlespacing{\subsubsection}{0mm}{-0.5em}{-0.3em}

\setlength{\parskip}{0.5em} % 1ex plus 0.5ex minus 0.2ex}
\setlength{\parindent}{0.5em}
\setlength\bibitemsep{\baselineskip}
% \setlength{\arrayrulewidth}{0.7mm}
\setlength{\tabcolsep}{3pt}

% \renewcommand{\arraystretch}{1.}
\def\arraystretch{1.5}%  1 is the default, change whatever you need
\emergencystretch 3em %reduce overhanging words
\renewcommand{\labelitemi}{{\color{fom}\textbullet}}

\counterwithout*{footnote}{chapter} %Fortlaufende Nummerierung für Fußnoten
\setcounter{secnumdepth}{5} % seting level of numbering (default for "report" is 3). With ''-1'' you have non number also for chapter

\AtEndDocument{\thispagestyle{empty}} %Letzte Seite ohne Header und Footer

% \lstset{
%          basicstyle=\footnotesize\ttfamily, % Standardschrift
%          numbers=left,               % Ort der Zeilennummern
%          numberstyle=\tiny,          % Stil der Zeilennummern
%          %stepnumber=2,               % Abstand zwischen den Zeilennummern
%          numbersep=5pt,              % Abstand der Nummern zum Text
%          tabsize=2,                  % Groesse von Tabs
%          extendedchars=true,         %
%          breaklines=true,            % Zeilen werden Umgebrochen
%          keywordstyle=\color{red},
%             frame=b,
%  %        keywordstyle=[1]\textbf,    % Stil der Keywords
%  %        keywordstyle=[2]\textbf,    %
%  %        keywordstyle=[3]\textbf,    %
%  %        keywordstyle=[4]\textbf,   \sqrt{\sqrt{}} %
%          stringstyle=\color{white}\ttfamily, % Farbe der String
%          showspaces=false,           % Leerzeichen anzeigen ?
%          showtabs=false,             % Tabs anzeigen ?
%          xleftmargin=17pt,
%          framexleftmargin=17pt,
%          framexrightmargin=5pt,
%          framexbottommargin=4pt,
%          %backgroundcolor=\color{lightgray},
%          showstringspaces=false      % Leerzeichen in Strings anzeigen ?
%  }
%  \lstloadlanguages{% Check Dokumentation for further languages ...
%          %[Visual]Basic
%          %Pascal
%          %C
%          %C++
%          %XML
%          %HTML
%          python
% 				%  json
%         %  Java
%  }

\lstset{
  basicstyle=\ttfamily,
  numbers=left,               % Ort der Zeilennummern
  columns=fullflexible,
  frame=single,
  breaklines=true,
  postbreak=\mbox{\textcolor{red}{$\hookrightarrow$}\space},
}

%\DeclareCaptionFont{white}{\color{white}}
%\DeclareCaptionFormat{listing}{\colorbox{gray}{\parbox{\textwidth}{#1#2#3}}}
%\captionsetup[lstlisting]{format=listing,labelfont=white,textfont=white}
%\renewcommand\lstlistlistingname{Listings}

\newcolumntype{Y}{>{\centering\arraybackslash}X}
\newcommand\RotText[1]{\rotatebox{90}{\parbox{3.5cm}{\centering#1}}}

\renewcommand{\appendixtocname}{Anhang}
